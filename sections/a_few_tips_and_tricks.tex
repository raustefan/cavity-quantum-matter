\section{A few tips and tricks}
\begin{itemize}
    \item Mount the cavity in the correct orientation, only one side can be used to couple in the light (see Fig.~\ref{fig:setup}).
    \item Use the beam profiler to measure the incident light. We used a diameter of around 180~$\mu$m.
    \item Do not use too much power, as the PD can not handle much. It saturates at around 1.8~V and dies at higher input powers.
    \item Use red light to visually align the cavity. The cavity works best if it is at a slight angle (also see Fig.~\ref{fig:setup}).
    \item The rough alignment should be done with visible light before switching to 1064~nm.
    \item A scan frequency of around 6~Hz is a good starting point for the cavity scan, as lower frequencies can cause noise in the signal.
\end{itemize}
